
\documentclass[a4paper,10pt]{article}

\usepackage{graphicx}
\usepackage[utf8]{inputenc}
\usepackage[T1]{fontenc}

\pdfmapfile{+ubuntu-regular.map}
\pdfmapfile{+ubuntu-it.map}
\pdfmapfile{+ubuntu-bold.map}
\renewcommand{\rmdefault}{Ubuntu}

%\usepackage{palatino}
%\usepackage{times}
\usepackage[top=2cm, bottom=2cm, left=3cm, right=5cm, marginparwidth=3cm]{geometry}
\usepackage{hyperref}

\usepackage[footnote,marginclue, draft]{fixme}


\newcommand{\ie}{{\textit{i.e.\ }}}
\newcommand{\cf}{{\textit{cf\ }}}
\newcommand{\eg}{{\textit{e.g.\ }}}

\graphicspath{{images/}}

%%%%%%%%%%%%%%%%%%%%%%%%%%%%%%%%%%%%%%%%%%%%%%%%%%%%%%%%%%%%%%%%%%%%%%%%%%%%%%%%%%%%%%%%%%%%%%%%%%%%%%%
%%%%%%%%%%%%%%%%%%%%%%%%%%%%%%%%%%%%%%%%%%%%%%%%%%%%%%%%%%%%%%%%%%%%%%%%%%%%%%%%%%%%%%%%%%%%%%%%%%%%%%%
\title{
    Starting with software development at CHILI
}
\author{}
%%%%%%%%%%%%%%%%%%%%%%%%%%%%%%%%%%%%%%%%%%%%%%%%%%%%%%%%%%%%%%%%%%%%%%%%%%%%%%%%%%%%%%%%%%%%%%%%%%%%%%%
%%%%%%%%%%%%%%%%%%%%%%%%%%%%%%%%%%%%%%%%%%%%%%%%%%%%%%%%%%%%%%%%%%%%%%%%%%%%%%%%%%%%%%%%%%%%%%%%%%%%%%%
\begin{document}
\pagenumbering{gobble}
\maketitle

\section*{Programming environments}

At CHILI, people are free to use the environment they are the most confident
with, be it C\# on Windows, Python on Linux, C++ on MacOSX, or any other
combination you prefer.

However, keep in mind a few important points:

\begin{itemize}

    \item we work as a team, and we share a lot of code: try as much as
        possible to rely on \emph{cross-platform} languages and libraries.

    \item in particular, we recommend you to rely on build systems like {\tt
        cmake} to ensure your code will be easy to compile on other platforms.

\end{itemize}

Examples of good cross-platform libraries you may want to consider before
starting a new project include \href{http://qt-project.org/}{Qt} (for GUI),
\href{http://www.mono-project.com}{Mono} (for cross-device development),
\href{http://opencv.org/}{OpenCV} (for image processing): an open-source
library exist for virtually every task, just ask for advices around you!

\section*{Code repositories and data storage}

During your time at CHILI, you are likely to produce some (or maybe a lot) of
code. To ensure your work will not be lost, we recommend you:

\marginpar{If you do not know GIT yet, do not worry: just ask around! People will
show you how it works.}

\begin{itemize}

    \item to rely on a Version Control System to manage your source code. At
        CHILI, we use \href{http://git-scm.org}{GIT} which is officially
        supported by EPFL. Check \url{git.epfl.ch}.

    \item large datasets (like videos) are better stored on our dedicated
        server: \url{smb://scxcraft.epfl.ch}. Ask your supervisor.

\end{itemize}

\vfill
\center{\bf And remember: ask around you (your supervisor, the PhD
students...)! We will be happy to help you to start!}

%%%%%%%%%%%%%%%%%%%%%%%%%%%%%%%%%%%%%%%%%%%%%%%%%%%%%%%%%%%%%%%%%%%%%%%%%%%%%%%%%%%%%%%%%%%%%%%%%%%%%%%

\end{document}


